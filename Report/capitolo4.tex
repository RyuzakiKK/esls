\chapter{Testing}

%TODO: scrivere qualcosa

\section{Testing della parte embedded}

\subsection{Unit tests}

Per quanto riguarda gli unit tests si è cercato di coprire le funzioni più complesse del sistema.

In tutti i test è stato utilizzato il framework \textit{unittest} di Python, il quale è stato realizzato inizialmente prendendo spunto da JUnit.

\subsubsection{Main Sanity Check \label{msc}}

In questo test si è controllata la correttezza della funzione \textit{sanity\_check} presente nel \textit{main}.

Quando le policy di un lampione vengono modificate viene chiamata la funzione \textit{sanity\_check} che ha il compito di assicurare che i dati inseriti siano validi e corretti.

I controlli che esegue passano dai più intuitivi come il fatto che l'orario debba essere nell'intervallo numerico 0-23 fino a controlli più complessi, come la verifica che il risparmio energetico sia compreso tra l'inizio e la fine delle policy di accensione e spegnimento.

Essendoci diversi casi limite come l'inizio delle policy in un giorno e la fine nel seguente, scrivere un unit test è stato di notevole aiuto per coprire diverse possibilità che si potrebbero presentare nella realtà.

In questo caso potendo essere influente anche il giorno e l'orario in cui si effettua il controllo, si è utilizzata la funzione \textit{patch} offerta da \textit{unittest.mock} per modificare a piacimento il valore relativo alla data corrente.


\subsubsection{Lamp Unit tests}

Sulla classe \textit{lamp} sono stati testati i due metodi statici \textit{get\_delta} e \textit{get\_wait}.

Il metodo \textit{get\_delta} prende come parametri un orario di inizio e uno di fine, e restituisce i secondi di differenza tra i due orari.

Per questo tipo di test si è provato un orario di inizio uguale alla fine, un orario di fine maggiore di quello di partenza, quindi relativo allo stesso giorno, e con un orario di fine minore, quindi relativo al giorno seguente.

Il metodo \textit{get\_wait}, in modo simile a \textit{get\_delta}, prende come parametri un orario di inizio e uno di fine, e restituisce i secondi di attesa necessari all'inizio dello scheduling.
Se l'orario attuale è già all'interno dello scheduling, verrà restituito un valore negativo corrispondente al numero di secondi già trascorsi dall'inizio.

In questo caso sono stati effettuati diversi test combinando i possibili orari di inizio, fine e l'orario in cui viene effettuata la richiesta.
Per modificare l'orario della richiesta è stata usata la funzione \textit{patch} come nel caso \ref{msc}.


\subsection{Integration tests}

%TODO: scrivere qualcosa

\subsubsection{Classe RPiMockGPIO}

Per simulare i pin di GPIO del Raspberry Pi è stata creata la classe RPiMockGPIO in modo da permettere di simulare l'esecuzione del programma anche senza l'utilizzo fisico di un Raspberry.

La funzione di GPIO più utile che viene utilizzata nel programma si chiama \textit{wait\_for\_edge} e consiste nell'aspettare che un determinato pin cambi stato.
Nel progetto viene utilizzata tutte le volte che si vuole misurare la distanza con il sensore ad ultrasuoni. Tenendo conto del tempo trascorso tra l'inizio e la fine dell'esecuzione di \textit{wait\_for\_edge} è possibile calcolare la distanza dell'oggetto in prossimità del sensore.

L'implementazione di \textit{wait\_for\_edge} nella classe RPiMockGPIO prevede semplicemente un \textit{time.sleep} su una variabile \textit{wait\_seconds}.
Per questo motivo regolando il valore di \textit{wait\_seconds} è possibile simulare diversi scenari. Impostando un valore alto (come ad esempio 1 secondo) si può creare una situazione in cui non ci siano mezzi in transito sulle carreggiate.
Al contrario impostando un valore basso (come 0.001 secondi) si potrà rilevare sempre auto in transito.

\subsubsection{Schedule test \label{st}}

Questo test permette di controllare la correttezza durante le operazioni di inizializzazione del sistema, accensione dei lampioni quando si è nella fascia di schedule e successivo spegnimento alla fine dello schedule.

La configurazione del test prevede:
\begin{itemize}
	\item La lettura del file \textit{cfg} di un lampione
	\item Il mock dell'orario corrente a 10 secondi prima dell'inizio dello schedule
	\item L'impostazione a 1 secondo della variabile \textit{wait\_seconds}, quindi corrispondete a nessuna auto in transito
\end{itemize}

A questo punto viene eseguito il \textit{main} del programma e dopo pochi secondi si controlla che il lampione che si sta testando sia spento, perché fuori dal suo orario di schedule.

Avendo impostato inizialmente l'orario corrente a 10 secondi dall'inizio dello schedule, il sistema aspetta qualche secondo e poi controlla che il lampione sia acceso come da programmazione.
La sua intensità deve corrispondere a quella impostata sul file \textit{cfg}.

Infine si imposta l'ora corrente 1 minuto dopo la fine dello schedule e si verifica il corretto spegnimento del lampione.

\subsubsection{Test rilevamento auto}

In questo test viene controllata la correttezza durante le operazioni di inizializzazione del sistema, rilevamento di mezzi in transito sulla carreggiata, ritorno a luminosità ridotta quando non sono presenti auto e infine spegnimento alla fine dello schedule.

La configurazione del test prevede:
\begin{itemize}
	\item La lettura del file \textit{cfg} di un lampione
	\item Il mock dell'orario corrente all'interno dello schedule per il risparmio energetico
	\item L'impostazione a 0 secondi della variabile \textit{wait\_seconds} permettendo di rilevare sempre mezzi in transito sulla carreggiata
\end{itemize}

In seguito viene eseguito il \textit{main} e dopo pochi secondi si verifica che il lampione sia acceso perché il sistema deve avere rilevato automobili sulla carreggiata.

Successivamente si imposta la variabile \textit{wait\_seconds} a 1 secondo, quindi in modo che non vengano più rilevate auto.
Dopo qualche secondo di attesa si controlla che l'intensità del lampione corrisponda al suo valore di risparmio energetico.

Infine, come per il test \ref{st}, viene impostata l'ora corrente 1 minuto dopo la fine dello schedule e si controlla che il lampione si sia spento correttamente.
