\chapter{Conclusione} 

Si ritiene che il sistema realizzato nell'ambito di questo progetto sia riuscito nell'intento di offrire un modo di risparmiare energia senza compromettere l'esperienza degli utenti del servizio.
\\Tutti i requisiti inizialmente specificati sono stati soddisfatti. Il sistema è stato progettato in modo da essere altamente flessibile ed estendibile in base alle esigenze specifiche di ogni città.
\\L'implementazione proposta non è la sola possibile: poiché ogni componente è stato pensato per essere modulare, è possbile scriverne la propria versione e utilizzarla in alternativa a quella proposta, purché siano rispettate le interfacce di comunicazione REST.
In particolare, per quanto riguarda la parte embedded, sebbene il sistema sia stato pensato per utilizzare Raspberry Pi, è del tutto possibile utilizzare altre piattaforme, in quanto il software per Raspberry è realizzato in Python. Lo stesso vale per la servlet, implementata in Java.
Sono possibili anche soluzioni basate su microcontrollori, sebbene in questo caso sarebbe necessario riscrivere il software della parte Raspberry.
\\A causa delle dimensioni del progetto, alcune funzionalità secondarie proposte in fase di progettazione sono state implementate solo parzialmente. Sono tuttavia supportate e implementabili agevolmente, in quanto già presenti nell'architettura.
\\Come sviluppi futuri, è plausibile un'implementazione di tali funzionalità e la realizzazione dei vari componenti su piattaforme hardware e software diverse, fornendo supporto anche ad installazioni eterogenee (ad esempio costituite sia da Raspberry, sia da altri single board computer, sia da microcontrollori).
In questo modo aumenterebbe il livello di flessiblità dell'intero sistema, rendendo ancora più agevole potenziare le installazioni esistenti, essendo il sistema più adattabile ad ogni scenario.
\begin{figure}[b]
	\centering
	\includegraphics[scale=1.0]{figure/cc-by-nc.png}
\end{figure}
