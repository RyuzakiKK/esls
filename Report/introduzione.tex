\chapter{Introduzione}

Grazie alle potenzialità delle nuove tecnologie viviamo in un mondo sempre più connesso, aprendo la porta alle smart city e alle loro potenzialità.
Nell'ambito di questo progetto si è approfondito il tema dell'illuminazione pubblica, un servizio pervasivo sul territorio e che comporta notevoli costi per i comuni, specialmente per quanto riguarda la spesa per l'energia elettrica.
\\L'introduzione delle lampadine a led sta consentendo di ridurre notevolmente tali costi, ma ci si è chiesti se sia possibile contenerli ulteriormente attraverso un uso più efficiente nel tempo.
In particolare sono state osservate numerose inefficienze per quanto riguarda gli orari di accensione e spegnimento, spesso pianificati a priori con dei semplici timer, senza tenere conto del progressivo spostamento degli orari di alba e tramonto di ogni singolo giorno dell'anno e delle condizioni meteo che possono ridurre la visibilità (temporali, ecc.).
I timer vengono effettivamente regolati per periodi diversi dell'anno, ma non con il livello di granularità ottimale, in quanto per modificare gli orari impostati è necessario l'intervento manuale di un addetto.
Inoltre, non tutte le zone di una città necessitano dello stesso livello di luminosità, sia a causa del differente utilizzo delle strade (strade principali, residenziali, del centro città, ecc.), sia a causa della specifica conformazione delle singole strade (ad esempio la presenza di alberi che ostacolano il passaggio della luce solare).
Infine, nelle ore notturne si può valutare una riduzione della normale luminosità dei lampioni, a causa del traffico estremamente ridotto.
\\Per questi motivi si è deciso di progettare e realizzare un sistema di gestione intelligente dell'illuminazione pubblica, che consenta l'amministrazione remota dei lampioni (controllati da dispositivi embedded), di regolare automaticamente la loro intensità luminosa al variare delle condizioni ambientali e di monitorare il loro funzionamento in maniera centralizzata, con lo scopo di permettere di prendere decisioni efficaci ai fini del risparmio energetico, ma senza compromettere la qualità del servizio offerto.
\\\\La relazione è strutturata nelle seguenti parti:
\begin{itemize}
 \item nel primo capitolo saranno definiti e analizzati i requisiti del sistema che si intende realizzare;
 \item nel secondo capitolo sarà descritta l'architettura del progetto, derivata dall'analisi dei requisiti, oltre allo schema di deployment;
 \item nel terzo capitolo verrà illustrata l'implementazione realizzata;
 \item nel quarto capitolo saranno presentati i test effettuati.
\end{itemize}
